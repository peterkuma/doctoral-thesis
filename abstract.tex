\chapter*{Abstract}
\addcontentsline{toc}{section}{Abstract}

\noindent
Southern Ocean (SO) shortwave (SW) radiation biases are a common
problem in contemporary general circulation models (GCMs), with most
models exhibiting a tendency to absorb too much incoming SW radiation.
These biases have been attributed to deficiencies in the representation of
clouds during the austral summer months, either due to cloud cover or cloud
albedo being too low. They affect simulation of New Zealand (NZ) and global
climate in GCMs due to excessive heating of the sea surface and the effect on
large-scale circulation. Therefore, improvement of GCMs is necessary for
accurate prediction of future NZ and global climate. Currently the New
Zealand Earth System Model (NZESM), based on the UK Hadley Centre
Coupled Model version 3 (HadGEM3), is developed at the National Institute
of Water and Atmospheric Research (NIWA) and the University of
Canterbury. We performed ship-based lidar, radar, radiosonde and weather
observations on two SO voyages and processed data from multiple past SO
voyages. We used the observations and satellite measurements for
evaluation of NZESM and contrasting with the MERRA-2 reanalysis to better
understand the source of the problem. Due to the nature of lidar
observations (the laser signal is quickly attenuated by clouds) they cannot be
used for 1:1 comparison with a model without using a lidar simulator, which
performs atmospheric radiative transfer calculations of the laser signal. We
modify an existing satellite lidar simulator present in the CFMIP
Observational Simulator Package (COSP) for use with the ground-based lidars
used in our observations by modifying the geometry of the radiative transfer
calculations, Mie and Rayleigh scattering of the laser signal. We document
and make the modified lidar simulator available to the scientific community
as part of a newly-developed lidar processing tool Automatic Lidar and
Ceilometer Framework (ALCF), which enables unbiased comparison between
lidar observations and models by performing calibration of lidar backscatter,
noise removal and consistent cloud detection. We apply the lidar simulator
on NZESM model fields. Significant SW radiation errors in the SO of up to 21
Wm$^{-2}$ are shown to be present in the model. Using the lidar observations,
we find that the model underestimates overall cloud cover by about 9\% and
strongly underestimates boundary layer low-level stratocumulus cloud below
1 km and fog. The observed cloud was strongly linked to the boundary layer
stability and sea surface surface, while this relationship is weaker in the
model. We identify that these errors are not due to misrepresentation of
large-scale circulation, which is prescribed in our model based on global
satellite observations by nudging. We conclude that the problem is likely in
the subgrid-scale parametrisation schemes of the boundary layer, convection
and large-scale could. In order to address the deficiencies identified we
perform experimental simulations of NZESM with modifications of the
parametrisation schemes. By comparing high-resolution output produced by
the lidar simulator with the lidar observations and model atmospheric
profiles with radiosonde observations, we study boundary layer processes
which lead to underestimation of stratocumulus cloud and fog in the model,
with the aim of fixing the schemes based on physically-motivated reasons
without negatively affecting global cloud simulation in the model.
