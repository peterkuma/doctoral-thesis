\chapter{Introduction}

Clouds represent one of the largest source of uncretainty in estimating
global climate sensitivity (K. D. Williams and Bodas-Salcedo 2017).
Clouds over ocean are especially important for determining the radiation
budget as the ocean surface has low albedo compared to land. Over the
Southern Ocean, cloud cover exceeds 80\%, with predominant boundary
layer clouds (Mace et al. 2009). Due to its large influence on
circulation and atmospheric transports in the Southern Hemisphere, the
Southern Ocean is imporant for global climate. Unlike most other places
on the globe, it is largely unaffected by sources of continental and
antropogenic aerosols, is dominated by a strong circumpolar vortex, and
has a southern boundary with a permanently ice-covered continent, which
may mean that global parametrisations do not apply well in this region.
Yet, observations in this region are sparse, which limits NWP (Numerical
Weather Prediction) and GCM (General Circulation Model) assimillation
and validation.

Shortwave (SW) radiation bias over the Southern Ocean of up to 30
Wm$^{-2}$ is a well-documented problem in current NWP models and GCMs
(Trenberth and Fasullo 2010), and it has been the subject of many
studies. It manifests both as a bias in SW radiation reaching the
surface and as shortwave reflectivity bias at ToA (Top of Atmosphere).
Bodas-Salcedo et al. (2014) looked at SW bias in a number of GCMs and a
strong SW bias is a very common feature, leading to increased SST (Sea
Surface Temperature) in the Southern Ocean. Trenberth and Fasullo (2010)
note that a poor representation of clouds might lead to unrealistic
projections for the Southern Hemisphere. The bias is linked to
large-scale model problems such the double-Intertropical Convergence
Zone (Hwang and Frierson 2013), position of the midlatitude jet and
meridional energy transport (Mason et al. 2014).

NZESM (New Zealand Earth System Model) is a new earth system model
derived from the UK climate model HadGEM/UKESM (Walters et al. 2017),
whose aim is to improve climate predictions for New Zealand, and
reducing Southern Ocean model biases is essential for achieving this
aim. The high-level UK model description by Walters et al. (2017) shows
that a clear and extensive SW radiation bias over the Southern Ocean is
presnet in the atmospheric component GA6.0 compared to satellite
radiation budget observations from CERES (Clouds and the Earth's Radiant
Energy System) (Wielicki et al. 1996). The bias in the context of the
Met Office models has been studied by A. Bodas-Salcedo et al. (2012) by
assessing cloud regimes in cyclones. Using observations from ISCCP,
MISR, MODIS, CALIPSO and CloudSat, they found that the model
understimates optical depth of stratocumulus and mid-topped clouds.
Recently, Davies, Loveridge, and Rampal (2017) studied boundary layer
clouds in the Southern Ocean compared to the Northern Hemisphere, with a
focus on supercooled liquid in clouds and cloud homogeneity. They note
that boundary layer clouds are the likely explanation for the bias due
to their large fractional coverage over the Soutern Ocean. Examination
of cloud cover in NZESM with the passive satellite instruments was
already performed by Schuddeboom et al. (2017) and is an ongoing effort.

While some authors focued on cloud distribution, others emphasise the
role of microphysics, especially supercooled liquid content. As
supercooled liquid has higher SW reflectivity than the equivalent amount
of ice particles, it has a positive effect on cloud albedo. Morrison,
Siems, and Manton (2011) studied the occurrence of supercooled liquid in
clouds over the Southern Ocean using observations from the MODIS
instruments and found it to be present year-round in low clouds at
temperature down to --40 $^{\circ}$C. R. P. Lawson and Gettelman (2014)
note that supercooled liquid is often underestimated in the Antarctic in
GCMs, and mixed clouds can occur at --32 $^{\circ}$C. They show that
increasing supercooled liquid in CESM (Community Earth System Model)
leads to CRE (Cloud Radiative Effect) increase of 7.4 Wm$^{-2}$ over
Antarctica. More recently, Kay et al. (2016) managed to fix the SW
radiation bias in CESM by increasing supercooled liquid in shallow
convective clouds, and notably they also needed to reduce a compensating
tropical SW radiation bias to maintain global radiation balance in the
model.

As reflected by the various hypothetised reasons for the SW radiation
bias by different authors, potential reasons for the observed SW
radiation bias can be numerous, concurrent and compensating. To
summarise, they can range from microphysical to large-scale dynamics and
be due to misrepresentation of:

\begin{itemize}
\itemsep1pt\parskip0pt\parsep0pt
\item
  Frequency of cloud regimes
\item
  Cloud vertical distribution
\item
  Cloud horizontal distribution (homogeneity) and overlap
  parametrisation
\item
  Cloud phase and supercooled liquid in clouds
\item
  Surface albedo (sea ice vs.~sea surface)
\item
  Moisture availability and transport
\item
  Large-scale circulation
\item
  Frequency of weather regimes
\item
  Frequency and scale of extratropical cyclones
\item
  Cloud regimes in cyclones
\item
  Cloud-aerosol interaction
\item
  Radiative transfer parametrisation
\item
  Boundary layer physics parametrisations
\item
  Dynamics -- meridional heat and moisture transport and large-scale
  convection
\item
  Cloud--temperature/humidity profile interaction
\end{itemize}

Multiple observations are available for assessing Southern Ocean biases.
Satellite observations provide the most complete record both spatially
and temporally (although they do not provide historical records prior to
approx. 1980s). They have been utilised by most studies of clouds in the
Southern Ocean and globally. Satellite instruments are very diverse,
though only a few datasets (satellite products) are readily available
for studying clouds. Operational geosynchronous satellites provide
near-continuous temporal coverage, which makes them ideal for studying
clouds, but they have limited use in high-latitude regions. In
combination with operational polar-orbiting satellites (POESS), they
have been used to produce a very long-term (1983--) cloud-oriented
dataset ISCCP (Schiffer and Rossow 1983). However, this dataset is
limited by a small number of spectral channels of the AVHRR instrument.
Other extensive cloud datasets include MODIS (on board of the A-Train
satellites Aqua and Terra) and APP-x (Extended AVHRR Polar Pathfinder)
(Meier et al. 1997). Recently, active instruments for observing clouds
became available: spaceborne radar on the CloudSat (G. L. Stephens et
al. 2008) and spaceborne lidar on the CALIPSO (D. M. Winker, Pelon, and
McCormick 2003) satellites, flying in a close coordination in the
A-Train satellite constellation. Together, they provide an unparalleled
view of global clouds due to their vertical view of clouds, generally
absent in passive-instrument observations. They have been utilised very
extensively in cloud studies globally. Other notable instruments
available for studying clouds include the multi-angle radiometer MISR
and passive microwave sensors, due to their ability to observe cloud
liquid water, total column water vapour, and vertically-resolved
temperature profile, and ability to `see' through clouds.

Passive satellite observations of clouds are ideal due to their high
spatial and temporal resolution, but have a number problems globally and
some specifically in polar latitudes (Bromwich et al. 2012):

\begin{itemize}
\itemsep1pt\parskip0pt\parsep0pt
\item
  Passive instruments can only observe upper-level clouds.
\item
  It is difficult to discern clouds from surface ice and snow in the
  visible spectrum due to similar albedo and in the IR (infrared)
  spectrum due to similar temperature and frequent inversion.
\item
  There is lack of sunlight during winter months, needed for SW
  measurements.
\item
  Poor detection of high-level thin clouds, falsely classified as
  mid-level clouds (J. M. Haynes et al. 2011).
\end{itemize}

Active satellite instruments are affected by signal attentuation in
thick clouds, and compared to passive instruments they have smaller
spatial coverage, shorter historical record and a small number of
spectral bands, limiting their ability to determine cloud microphysical
properties.

In addition to spaceborne observations, ground-based and in situ
observations can provide a vital complementing view of clouds `from
below' and `inside'. These include high-frequency radars, ceilometers
(lidars), pyranometers, sky cameras, radiosondes, dropsondes, in situ
aerosol measurements (cloud condensation nuclei and ice nuclei) and
airborne observations from drones, weather balloons and aircrafts. These
observations are logistically difficult and expensive, and are generally
sparse in the Southern Ocean, with limited time periods and limited
historical records. Use of ground-based and in situ observations alone
for assessment of GCMs is difficult due to their small
representativeness of climatic conditions.

Different processing of observations from the same instruments can lead
to different results (e.g. H Chepfer et al. (2010) in their GOCCP
product). For example, different thresholds can exists which define what
is `cloud', and cloud detection is affected by targeting a particular
false alarm ratio, such as 5\% as in the CloudSat-CALIPSO dataset
(Hagihara, Okamoto, and Yoshida 2010). Probability of detection
(sensitivity) then depends on the ROC (Receiver Operating
Characteristic) curve, which in turn depends on instrument errors.
Moreover, instrument errors can vary over lifetime of a single
instrument, or between instruments in multi-instrument datasets. A
problem with different processing algorithms was noted by Martucci,
Milroy, and O'Dowd (2010), who compared manufacturer-supplied cloud base
height determination from colocated Vaisala CL31 and Jenoptik CHM 15k
ceilometers and found a poor agreement, but developed a new algorithm
for determining cloud base height which leads to consistent heights
between the two instruments.

Due to the reasons outlined above, a combination of multiple passive,
active, ground-based and in situ observations are needed to
comprehensively assess cloud climatology and biases. This has been also
noted by other authors: K. D. Williams and Bodas-Salcedo (2017) assess
cloud representation in Met Office UM model using a multi-dataset and
multi-diagnostic approach, and stress the importance of using multiple
instruments due to compensating errors in GCMs. While use of single or
combined satellite observations to assess model performance is common in
many studies, combination of ground-based and spaceborne instruments is
less common. For example, Muhlbauer et al. (2015) studied cirrus clouds
using A-Train observations (CloudSat, CALIPSO, MODIS, CERES),
ground-based ARM (Atmospheric Radiation Measurement) radar and aircraft
observations. J. Zhang, Xia, and Chen (2017) performed a comparison of
satellite and ground-based cloud observations at an ARM site.

Comparison between models and observations cannot always be performed
directly, esp. if observations do not produce model quantities or
measure only a variable part of a model field (e.g.~cloud tops). Even
though observations can be mapped to model fields by inversion, this may
be unreliable due to a large number of factors involved and a limited
view of the instrument. Conversely, model fields can be mapped to
observations by instrument simulators, and this approach has been used
extensively in a number of studies. Satellite simulators such as COSP
(A. Bodas-Salcedo et al. 2011) solve the problem by transforming model
fields to observation fields. The COSP simulator was developed as part
of CFMIP (Cloud Feedback Model Intercomparison Project) (S. Bony et al.
2011), and it was used for evaluation of GCMs in CMIP5 (Coupled Model
Intercomparison Project Phase 5) (Taylor, Stouffer, and Meehl 2011). The
simulator implements a number of passive and active instruments:

\begin{itemize}
\itemsep1pt\parskip0pt\parsep0pt
\item
  ISCCP
\item
  MODIS
\item
  MISR
\item
  CloudSat
\item
  CALIPSO
\item
  ARM ground-based radar (MMCR/KAZR)
\item
  RTTOV
\end{itemize}

Notably, radar observations are simulated by the QuickBeam simulator
(Haynes et al. 2007) and lidar (CALIPSO) observations are simulated by
the ACTSIM simulator (H{é}lene Chepfer et al. 2008). In general, these
may need to be tuned for any particular instrument being simulated due
to different wavelengths, signal modulation, view and error
characteristics. The COSP simulator allows for comparison of instrument
quantities (backscatter, radar reflectivity), or derived products (cloud
top/base, cloud phase, \ldots{}) between the model and observations. An
exact colocated comparison is limited by the low spatial resolution of
GCMs, and pseudo-observations need to be made on subcolumns generated by
a cloud generator. Algorithms for calculating derived products are
generally not available, and datasets such as CALIPSO-GOCCP were
developed for the purpose of comparison of equivalent quantities from
observations and the simulator (H Chepfer et al. 2010).

In order to assess cloud representation in GCMs, it is imporant to
attribute biases in observations to model processes. However, these
often cannot be observed directly, and meaningful hypotheses about
deficiencies in these processes need to be formed and tested by
observations. It was noted by S. A. Klein and Jakob (1999) that an
observed bias can be due to a large number of causes in the model. Cloud
parametrisations in models include:

\begin{itemize}
\itemsep1pt\parskip0pt\parsep0pt
\item
  Cloud liquid and ice water content
\item
  Cloud droplet size distribution
\item
  Ice particle size and habit
\item
  Cloud distribution parametrisation (McICA)
\item
  Radiative transfer approximations in cloudy atmosphere
\item
  Supercooled liquid parametrisation in clouds
\item
  Cloud-aerosol interaction
\end{itemize}

Clouds are tightly linked to their effects on radiation balance. In the
NZESM model the radiative transfer parametrisation follow J. Edwards and
Slingo (1996; J. M. Edwards et al. 2015). Clouds are represented by
convective and stratiform liquid/ice mass mixing ratio and effective
radius. Subcolumn cloud distribution is treated by the McICA method
(Barker, Stephens, and Fu 1999), with a stochastic cloud subcolumn
generator based on gamma distributed cloud water content and vertical
overlap defined by a decorrelation depth (R{ä}is{ä}nen et al. 2004). It
is notable that some of the parameters are fixed globally, and as such
may be a poor representation for the southern polar latitudes,
especially considering the poor validation possibilities in this region.

This study will focus on complementing other studies assessing
representation of clouds, aerosols and cloud-aerosol interaction in the
Southern Ocean (taking into consideration the Southern Hemisphere and
global processes) in the NZESM, with a particular focus on utilising
ground-based measurements available from intensive observation periods
and transient sites. For this purpose the COSP simulator will need to be
extended to support these instruments. It is foreseen that the
ground-based measurements will need to be combined with satellite
observations, especially CloudSat and CALIPSO, but exploring
possibilities to use passive instruments (if they can provide
complementary information to other studies). Upscaling of ground-based
observations (outlined in the methodology section) will be explored as a
means of making ground-based observations more relavant for
characterising climate. Other diagnostic means may include case-studies,
calculation of climatologies and classification. Particular focus of the
study will be on linking observed biases to model processes, and
following closely the NZESM development process.

\section{Objectives}\label{objectives}

\begin{itemize}
\item
  Use satellite, ground-based and in situ observations to identify
  processes related to clouds and aerosols in the NZESM leading to
  biases in the Southern Ocean.
\item
  Extend the COSP simulator to be able to simulate ground-based lidar
  and radar observations available.
\item
  Use selected active and passive satellite instruments to assess NZESM
  results, complementing other past and ongoing research.
\item
  Use collocated satellite and ground-based observations in the Southern
  Ocean to assess their differences.
\item
  Explore potential for extrapolating (upscaling) ground-based
  observations by correlating them with satellite observations, e.g.~by
  classification of both and linking correlated classes.
\end{itemize}

\section{Methods}\label{methods}

Central to our study will be a dataset of ground-based observations from
a number of IOPs (Intensive Observation Periods) and transient sites in
the Southern Ocean and the Antarctic region:

\begin{itemize}
\itemsep1pt\parskip0pt\parsep0pt
\item
  Aurora Australis
\item
  Macquaire Island (2016--)
\item
  RV Tangaroa (2015), 2×
\item
  HMNZS Wellington (2016)
\item
  RV Tangaroa (14 March -- 1 April 2017)
\item
  Nathaniel B. Palmer (April -- June 2017)
\end{itemize}

This observations include measurements from ground-based ceilometers
(Vaisala CL51 and Lufft CHM 15k) and radar (Meter MRR-2), radiosondes,
aerosol particle counters and AWS (Automatic Weather Station).

These observations are complemented by publicly-available datasets from
comprehensive measurements at the ARM mobile facilities:

\begin{itemize}
\itemsep1pt\parskip0pt\parsep0pt
\item
  AWR M1 -- McMurdo, Antarctica (October 2015 -- September 2016)
\item
  MAR M1 (MARCUS) -- Ship voyage from Hobart, Australia to Antarctic
  coast on Aurora Australis (planned September 2017 -- April 2018)
\end{itemize}

Our assessment will focus on the NZESM model, although contrasting with
other GCMs may be useful. We will focus on biases in the Southern Ocean
and the Antarctic, but pay attention to any processes relavant to the
Southern Hemisphere and globally. A one-year 2007 simulation from a
nudged and un-nudged run of NZESM is already available, with more
simulations in progress.

We will utilise the COSP simulator to produce pseudo-observations
equivalent to available ground-based measurements. This will require
running the COSP simulator in the `offline' mode on fields output from
the model run, and development of new code specific to the instruments.
It is likely that the existing lidar and radar simulators ACTSIM and
QuickBeam can be reused for this purpose.

It is foreseen that the ground-based and in situ observations will not
provide enough information alone, and a combination with satellite
observations will be necessary. A possible technique of upscaling may
allows us to utilise these observations on a wider scale. Upscaling can
be performed by a linked classification of colocated ground-based and
satellite observations. Methods such as Bayesian classification (via
MCMC (Markov chain Monte Carlo) simulation), k-means clustering, SOM
(self-organising map) and CCA (canonical correlation analysis) can be
utilised.

Satallite datasets will include the active instruments on CloudSat and
CALIPSO, complemented by the CERES dataset for assessing radiation
budget, and possibly MODIS, ISCCP and APP-x datasets, exploring the
possibility to use passive microwave datasets for cloud water content
and moisture assessment.

K. D. Williams and Bodas-Salcedo (2017) summarised a number of
compositing techniques for cloud assessment with respect to specific
atmopheric conditions:

\begin{itemize}
\itemsep1pt\parskip0pt\parsep0pt
\item
  Large scale vertical velocity
\item
  Lower tropospheric stability
\item
  Position relative to cyclone centre
\item
  Cloud regime
\end{itemize}

These may be used to complement methods outlined above.

In our study we will try to assess specific deficiencies in the NZESM
parametrisations affecting clouds and radiative transfer, in order to
assess the relative importance of cloud macrophysical and microphysical
properties in the observed biases. This has been explored to some extent
by other authors, but not always in the context of the UK model or
NZESM, where the causes can be different. Some of the questions we may
try to answer are:

\begin{itemize}
\itemsep1pt\parskip0pt\parsep0pt
\item
  Is frequency of modelled cloud regimes consistent with observations?
\item
  Does the vertical distribution of clouds correspond to observations?
\item
  Are the chosen subcolumn cloud generator parameters suitable for the
  Soutern Ocean?
\item
  Is supercooled liquid well-represented?
\item
  Is modelled humidity and cloud liquid water consistent with passive
  microwave measurements?
\item
  Are modelled cloud bases consistent with those observed by a
  ceilometer?
\item
  Are clouds simulated well in different weather regimes liked to
  cyclones and anticyclones?
\item
  Is the sea ice albedo affecinting ToA reflectivity biases?
\end{itemize}

Adjacent to our study will be development of a publicly-available
dataset of ground-based and in situ observations in the Southern Ocean.

