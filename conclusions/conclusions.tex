\chapter{Conclusions and further work}

Our initial objectives were to conduct intensive observation campaigns (IOPs)
in the Southern Ocean (SO), collate and post-process existing and new
ground-based SO datasets, extend the Cloud Feedback Model Intercomparison
Project (CFMIP) Observation Simulator Package (COSP) lidar simulator with support for
ground-based lidars, evaluate SO cloud biases in the New Zealand Earth System
Model (NZESM) with the aim
of identifying processes responsible for the biases, and perform
experimental simulations with the aim of improving the cloud biases
(Sect. \ref{sec:objectives}). We have addressed these objectives by:

\begin{itemize}
\item conducting field measurements on the TAN1702 and TAN1802 voyages,
deploying our instruments on the 2016 HMNZS \textit{Wellington} and NBP1704 voyages,
collating, calibrating and post-processing data from the TAN1502, TAN1503,
2015--2016 \textit{Aurora
Australis} voyages and Macquarie Island (Chapter 2),
\item developing the Automatic Lidar and Ceilometer Framework (ALCF) 
based on the COSP lidar simulator (Chapter 3),
\item performing nudged simulations with the Global Atmosphere, version 7.1
(GA7.1) corresponding to the
time periods of observations and evaluating cloud representation in the model
and the Modern-Era Retrospective analysis for
Research and Applications, version 2 (MERRA-2) (Chapter 2),
\item performing an experimental simulation of the model with modified
boundary layer (BL) processes, which lead to an improvement of cloud representation
(Chapter 4).
\end{itemize}

\noindent
As a result, we have:

\begin{itemize}
\item developed a dataset of multi-voyage ground-based SO observations
(manuscript in preparation),
\item found that BL cloud is strongly implicated
in the model biases, BL cloud occurrence is underestimated in the GA7.1 and MERRA-2,
optical thickness of individual clouds is very likely overestimated, and cloud
base height is linked to the local thermodynamic conditions (Chapter 2),
\item developed an open source framework for automatic lidar and ceilometer
(ALC) data processing and comparison with models using a lidar simulator
(Chapter 3),
\item found that an experimental run of the GA7.1/UM11.4 with an increased 
mass flux and surface moisture flux leads to a better representation
of stratocumulus cloud in comparison with the TAN1802 voyage ceilometer
observations using the ALCF, but the flux between the surface mixed layer
and the convection layer appears to be a limiting factor for the improvement
of BL cloud.
\end{itemize}

To summarise, we showed that ground-based SO observations are a very useful
complementary dataset to satellite observations. They can reveal model cloud
biases not identifiable from space due to obscuration by higher-level cloud
(passive visible and infrared instruments and lidars) and ground clutter (radars).
Considering that the model cloud biases are predominantly in the BL, ground-based
observations appear to be more suitable for improving them. In this
sense SO voyage observations are very valuable in a region which is highly
inaccessible and where there are few ground-based observations both spatially
and temporally.
Ground-based
lidar processing and instrument simulator tools have so far been given
less attention than spaceborne instruments. A high diversity of these instruments,
different output formats, noise characteristics, calibration and a lack of openly-available
software are some of the problems hindering model evaluation using these
instruments. We made an effort to improve this situation by releasing
open source tools for converting instrument data (cl2nc, mrr2c, mpl2nc), already
in use by the scientific community, and the ALCF, which streamlines the
processing of lidar data and simulating lidar backscatter by supporting
common reanalyses and models (AMPS, ERA5, JRA-55, MERRA-2 and the UM).
We showed that a lidar simulator can be not only useful in a statistical
comparison, but also in a case study approach in combination with a nudged
model or a reanalysis, revealing deficiencies in the simulation of individual
clouds if the model output has a sufficiently high temporal resolution.
We showed that this approach can be used successfully for improving model 
parametrisations affecting BL cloud (turbulence and convection).
While our experimental run of GA7.1/UM11.4 showed an improvement of
stratocumulus cloud simulation, further work is needed before the suggested
modifications can be reliably applied in the operational model. In particular,
the connection between the BL and convection schemes appears to be too weak,
and may need to be revised.

\section{Further work}

Even though we analysed a large part of the ground-based SO observations
available to us (Chapter 2), they are still underutilised. For example,
we haven't analysed the dual-polarisation lidar data collected on TAN1802
by a MiniMPL. If properly calibrated, these could provide crucial information
about supercooled liquid cloud. The micro rain radar (MRR-2) deployments
have so far been underutilised. They allow for detection of precipitation,
rain rate and vertical velocity in precipitation. Model evaluation of
precipitation is needed to complement cloud evaluation in order to make
sure the right amount of moisture is being removed from the BL by precipitation
and the precipitation has the correct phase (liquid or ice). Similar to clouds,
precipitation has an effect on shortwave (SW) and longwave (LW) radiation. It is therefore important
for the radiation balance, considering its frequent occurrence in the SO.
A dual camera setup was installed on the TAN1802 voyage in order to allow
for stereoscopic determination of cloud base height. This imagery hasn't been
utilised so far. Co-location with the ceilometer observations provides an
opportunity for algorithm development and cross-validation of results.
Likewise, cloud type and cloud fraction can be determined from sky camera
imagery using a suitable algorithm. A pair of sky cameras could be utilised as
a low-cost instrument for determining cloud base height, cloud fraction
and cloud type on ships of opportunity.

As we identified, ground-based observations have a large additional value
to satellite observations when it comes to BL cloud evaluation. More
ground-based observations in the SO could narrow the spatial and temporal gaps
in our understanding of the region. Therefore, making ship-based atmospheric
observations more accessible is an important task. Currently, deployment
of instruments such as lidar and radars on ships is logistically difficult
and expensive. Smaller and less expensive instruments could mean that they
can be deployed more widely and installed permanently on some ships. Smaller
lidars typically have lower power and inferior noise characteristics,
but a wide coverage could outweigh these deficiencies. Progress in lidar
development could also mean smaller instruments may become equally powerful.
The commercial nature of common instruments means that some instruments
suffer from a number of technical problems and vendors typically show
little interest in resolving these. Open hardware and open software design of
instruments could substantially accelerate progress and collaboration.
Off-the-shelf software defined radio (SDR) receivers and
transceivers\footnote{\url{https://limemicro.com}.} have recently
become widely available. They could provide a basis for development of improved
radars which can utilise many frequencies. Open software for processing
instrument data could mean that implementation of standard techniques and
algorithms for calibration, resampling, noise removal and calculation of derived
products become more available to the scientific community. Likewise,
public sharing of ground-based observational data and building of dataset
collections, similar to the common practice of releasing satellite datasets,
has a potential to accelerate atmospheric research in the SO.
Utilising unusual platforms such as the Argo floats \citep{roemmich2009}
for atmospheric measurements could provide a vast amount of data and a relatively dense
coverage in the SO.
Air--sea surface fluxes appear to be implicated in the GA7.1/UM11.4 cloud biases
(Chapter 4). Yet, the Coupled Ocean–Atmosphere Response Experiment (COARE)
formulas are potentially not well-tuned for high latitudes and atmospheric
reanalyses exhibit large air--sea flux biases in the SO
\citep{cerovevcki2011}. Evaluation of model air--sea flux biases relative to
reliable in situ flux measurements is needed to make sure they are not
a major cause of the BL cloud biases.

A reliable comparison of models and observations using a lidar simulator
requires accurate absolute calibration of the observed backscatter
(Chapter 3). We have identified various deficiencies in the vendor-supplied
calibration such as the dead time, afterpulse and overlap calibration in
the MiniMPL. These often result in range-depended bias of the volume
backscattering coefficient, which may be impossible to reliable correct
for without dedicated calibration measurements. Currently, the ALCF only
supports height-independent calibration utilising observations of liquid
stratocumulus clouds and comparison with a theoretically expected
molecular backscatter. An automated approach should be developed to calculate
dead time, afterpulse and overlap corrections as an alternative to the 
vendor corrections, which appear to be unreliable. This could lower the
difficulty of utilising ceilometer and lidar data by the wider scientific
community, which is a significant barrier for a wide utilisation of ceilometers
and lidars for model evaluation.
Currently, the ALCF does not support calculation of cross-polarised
backscatter. Adding support for simulation of cross-polarised backscatter
would mean that both channels of the Sigma Space MiniMPL can be utilised
in a comparison with a model and therefore evaluate cloud phase.
Likewise, aerosol and precipitation is currently not taken into account
by the lidar simulator. This not only causes a systematic bias in comparison
with observations, but also prevents the simulator to be used for
model aerosol and precipitation evaluation.

A large amount of automatic lidar and ceilometer (ALC) data have been collected by regional and global
networks such as Cloudnet \citep{illingworth2007},
E-PROFILE \citep{illingworth2018}, EARLINET
\citep{pappalardo2014}, ICENET \citep{cazorla2017} and MPLNET
\citep{welton2006}. Most of these networks utilise ceilometers
and lidars already supported by the ALCF such as Vaisala CL31, CL51, Lufft CHM
15k and Sigma Space MiniMPL. These observations haven't been used
extensively for model evaluation. There is a potential to process and compare
these observations with models using the ALCF. Such comparison would complement
past comparisons with satellite observations, and could rival satellite
observations in terms of global spatial and temporal coverage. 

Here, we utilised the Clouds and the Earth's Radiant Energy System (CERES) satellite observations of top of atmosphere (TOA) radiation. Even
though the errors of these observations may be relatively low compared to 
current model cloud biases, systematic errors in the dataset exist due
to unequal diurnal sampling and the need to for temporal and angle
interpolation of the raw measurements to calculate daily averages. 
The relatively new instrument National Institute of Standards and Technology
Advanced Radiometer (NISTAR) on the L1-stationed Deep Space Climate Observatory (DSCOVR)
provides a unique viewpoint for Earth radiation observations, from which 
the sunlit part of the Earth, including polar latitudes, is visible continuously.
It would be valuable to compare model radiation fluxes with these new
observations to rule out the effect of systematic biases of CERES on the
model TOA radiation evaluation.

An unresolved question about the origin of the bipolar SW radiation
bias in the high- and low-latitude SO in GA7.1 remains. We know that
the model underestimates low cloud, and at the same time, we identified
some evidence that the optical thickness individual clouds is overestimated.
This may be the underlying reason for the overestimated reflected TOA SW
radiation in the low-latitude SO.

We identified that BL parametrisation of mass flux can be tuned
to enable stratocumulus cloud simulation in the SO (Chapter 4). However,
a good observational reference for mass flux and the similarity relationships
is lacking in the SO. The BL turbulence and convection
parametrisation is based on large eddy simulations (LES) initialised
from field experiments in the tropical and midlatitude ocean. These may be
inappropriate for the SO. Therefore, new LES initialised from SO field
experiments are needed to make sure these parametrisations are correct in this
region.

We have shown that nudged simulations of GA7.1 provide a very good basis
for identifying model biases compared to ground-based observations. As opposed
to a free-running model, it can be reasonably assumed that any biases are
not the result of a different weather situation simulated by the model
than in reality, and that they are largely due to errors in the subgrid-scale
parametrisation processes not assimilated in the ERA-Interim reanalysis driving
the model or not an input to the model nudging algorithm. As shown in Chapter 4,
a side-by-side comparison of modelled and observed cloud is feasible, and this
can be utilised in future studies of model clouds.

In Chapter 2 we presented a dataset of ground-based observations in the SO.
This dataset provides a unique and comprehensive information on SO atmospheric
conditions and clouds. Work is underway to make this dataset documented and
publicly available. To this end, derived products should be developed in order
to make it easy for the scientific community to use this dataset. In general,
a more concentrated effort is needed to streamline public sharing of atmospheric
observations, especially considering the global and accelerating
effect of climate change, which has been called
"the defining challenge of our time" \citep{wmo2019} by the United Nations
Secretary-General A. Guterres. It is the author's opinion that the
seriousness and urgency of the situation is vastly underestimated even
by the atmospheric science community, which continues to hinder international
cooperation by not sharing data and model code, and by publishing scientific
results in paywalled journals, and therefore putting the well-being of future
generations in jeopardy.

In Chapter 4 we identified that increasing convective mass flux and surface
heat flux can improve stratocumulus cloud simulation. However, it remains
to be proven if mass flux or surface fluxes are underestimated relative to
a physical reference (either an observational reference or large eddy
simulations). More model experiments need to be performed which increase
flux between the turbulently-mixed surface mixed layer and the convective layer,
as well as longer term climate simulation to make sure the modifications
address the actual problem and they do not have a negative impact on the
global radiation balance throughout the year. A new BL scheme "CoMorph"
is currently in development at the UK Met Office. Our findings could contribute
to the development of this new scheme, but more experiments need to be performed
with this new scheme and comparison of this new scheme with the ground-based
SO observations should be performed.

\clearpage
