\chapter{Conclusions and further work}

We conducted ship-based observations on the 2-week long voyage TAN1702 of
RV \textit{Tangaroa} to the Campbell Plateau
in March 2017 and the 6-week long voyage TAN1802 of RV \textit{Tangaroa} to the Ross Sea
in February--March 2018. These included ceilometer, lidar, radar, radiosonde,
sky camera, automatic and human weather observations and aerosol measurements.
We deployed ceilometers, radar and a sky cameras on additional voyages:
the December 2016 voyages of HMNZS \textit{Wellington} and an 
April--May voyage NBP1704 of RV \textit{Nathaniel B. Palmer} to the Ross Sea.
In addition we collated and processed observations from previous voyages
of \textit{Aurora Australis}, TAN1502, TAN1503 and land-based ceilometer
observations on Macquarie Island.
We set up and performed nudged simulations
of the Global Atmosphere version 7.1 (GA7.1) on a supercomputer of the New Zealand eScience Infrastructure (NeSI) and the National Institute of Water \& Atmospheric Research (NIWA).
We compared the model output with our collection of ship-based observations,
focusing on geometrical properties of clouds observed by ceilometers and their
links to thermodynamical profiles of the atmosphere observed with radiosondes.
We contrasted these with the MERRA-2 reanalysis. Both the nudged GA7.1 and MERRA-2
showed significant cloud occurrence biases of 4--18\% less cloud than in
ceilometer observations. Despite of this, both showed a bipolar shortwave (SW)
radiation
bias by latitude, with outgoing SW radiation underestimated in high-latitude
SO and overestimated in low-latitude SO. This suggests that compensating biases
of cloud occurrence vs. cloud optical thickness are present in the models.
This bipolar SW radiation bias was strongly correlated with near-surface air temperature,
with relatively low air temperature regions exhibiting a negative SW radiation bias and
relatively high temperature regions exhibiting a positive SW radiation bias. By analysing
lifting levels, lifting condensation level (LCL) and the level of neutral buoyancy
of an air parcel with potential temperature equal to sea surface temperature (SST),
we showed that cloud base in the region is strongly linked to these levels,
and these levels were a better predictor for cloud base than the previously
utilised lower tropospheric stability (LTS). This finding suggests that local
thermodynamics is a strong driver of the summertime SO stratocumulus (Sc) clouds,
a finding also supported by \cite{hartery2020b}. Interestingly, we found
very large differences in cloud phase simulated by the nudged GA7.1 and MERRA-2.
MERRA-2 simulated much greater amount of liquid cloud (the majority of
the water path), while the GA7.1 simulated slightly more ice than liquid.
Even though we did not use an observational reference for cloud phase,
we can say that the cloud phase was most likely behind the difference in
cloud biases between the GA7.1 and MERRA-2 (in addition to the identified cloud occurrence biases), which were much more positive
in MERRA-2 (liquid cloud reflects more SW radiation for the same amount of water
path). We conclude that cloud geometry biases play at least an equally
significant role in radiation biases as cloud phase and these two can be
compensating biases, a fact which was not given enough recognition in previous
studies. The source of the bipolar biases and its strong relationship with
near-surface temperature, however, is still relatively unclear.

In order to compare ceilometer and lidar observations with models,
we developed the Automatic Lidar and Ceilometer Framework (ALCF),
a tool which extends the Cloud Feedback Model Intercomparison Project (CFMIP) Observation Simulator Package (COSP) lidar simulator with support for a range
of ground-based instruments and a processing pipeline which allows
for an unbiased comparison of simulated and observed backscatter and a cloud mask.
This tool was the subject of Chapter 3. While the COSP lidar simulator
provided a basis for the physical simulation of laser radiation transfer
through the atmosphere, more work needed to be done to support the ground-based
instruments with their variety of field of view (FOV), wavelengths and
calibration issues. By developing and documenting this tool, we enabled future
studies utilising off-the-shelf ground-based lidars for model evaluation.
The potential of ground-based lidars has so far been underutilised compared
to similar active satellite observations (CALIPSO) due to the
lack of software processing tools and the difficulty of accurate calibration
of these instruments. The problems addressed by our work included Mie
scattering at different laser wavelengths depending on droplet size distribution
parameters, absolute calibration by utilising Sc clouds and molecular
backscattering, backscatter noise removal, cloud mask/cloud base determination 
and evaluation of noise characteristics of various common lidars. We demonstrated
usefulness of this new tool on several case studies. Several common atmospheric
models and reanalyses are supported by this framework, which allowed us to compare
a range of publicly-available model output with ceilometer and lidar observations
at several stations in Aotearoa/New Zealand. We showed how this tool can be used
for comparing vertical cloud occurrence, backscatter "curtain" plots
and cloud fraction vs. optical thickness. More processing algorithms can
be incorporated in this modular tool in the future to allow for calculation of
derived products.

Lastly, we focused on evaluation of SO boundary layer cloud in the GA7.1
and how the BL turbulence and convection parametrisations affect
BL cloud. Previously, we found that Sc cloud below 2 km above sea level (ASL)
and fog were predominant in the SO, but underestimated in the model. We used
the ceilometer observations collected on TAN1802 to evaluate the model cloud
in a case study based approach, where we compared "curtain" backscatter plots
between the model and observations using the ALCF. This allowed us to identify
in detail that the model is missing the extensive layers of Sc cloud found
in observations, while overestimating cumulus (Cu) clouds occurring below the
Sc layers. In observations, these layers were found to correspond to LCL (Cu clouds),
and the level of dry and moist neutral buoyancy of a parcel with potential temperature
equal to SST lifted from the sea surface (Sc clouds). Therefore, BL
thermodynamics was identified as being key to resolving these biases. By running
a model experiment with an increased surface flux and mass flux, we tested
a hypothesis that not enough moisture transport is simulated to accumulate
moisture at the top of the BL for cloud condensation. When compared with
observations, this experiment lead to an improved occurrence of Sc clouds,
but due to a limiting factor we were unable to attain a full correspondence
with observations. We identified the coupling between the turbulence and convection
parametrisations across LCL as the most likely bottleneck. Therefore, it seems
that the separation of parametrisation into turbulence and convection driven
vertical regions appears to limit the formation of Sc clouds in the BL in the
summertime SO. Our experimental run showed promising results in terms of
SW radiation biases globally (evaluated on a short monthly time period in February--March 2018),
where the zonal means were reduced by up to 5 Wm$^{-2}$. We conclude that
further modifications to the BL schemes are needed to improve
Sc cloud simulation in the SO. The tuning in the experimental run also needs to
be evaluated globally over a long time period to determine if it can be used
operationally. This approach when individual clouds can be compared between
a nudged model and observations means that compensating biases are avoided.
Secondary to this approach should be evaluation of cloud optical thickness,
cloud phase and the cloud--aerosol effects, which all contribute to the
radiation biases. In this sense, our work is complementary to that of
\cite{revell2019}, \cite{hartery2020a} and \cite{hartery2020b}.

We can summarise our contributions as follows. The relatively large SO
cloud biases and the resulting radiation biases are still common in GCMs
and reanalyses today \citep{gettelman2020}. Due to the relatively large
extent of the SO, its role in the uptake of excess heat and CO$_2$
and being a buffer zone for warming of the Antarctic ice sheets, and the Southern Hemisphere
meridional circulation, it has a crucial role in the Earth's climate system.
Clouds are the major factor in Earth's albedo variability. Therefore, any cloud
biases can result in large biases in a model's large scale atmospheric
and oceanic circulation and the cryosphere. Unless these biases are minimised
in GCM simulations of the past and present climate, the accuracy of future climate
simulations is limited, especially when it comes to predicting the change
of cloud cover, the Earth's albedo, circulation and the ice sheets.
Using ship-based observations, we narrowed down the cloud climatology
in the SO, its main drivers and factors involved in the cloud biases. We
identified BL parametrisations of turbulence and convection as a major deficiency
in the model, and suggested concrete improvements. This work can translate into
eventually improving these parametrisations. The improved understanding of the
drivers can be applied in the development of other GCMs and reanalyses.
Development of the ALCF as a tool for evaluation of models using ground-based
lidar observations can streamline the process of using these type of observations
globally, and provide a unique opportunity to evaluate BL clouds, some of
which are not visible with spaceborne instruments. This new tool substantially
lowers the bar for utilising these instruments, and therefore it can enable
future studies of BL clouds.

\section{Further work}

Even though we analysed a large part of the ground-based SO observations
available to us (Chapter 2), they are still underutilised. For example,
we have not analysed the dual-polarisation lidar data collected on TAN1802
by a MiniMPL. If properly calibrated, these could provide crucial information
about supercooled liquid cloud. The micro rain radar (MRR-2) deployments
have so far been underutilised. They allow for detection of precipitation,
rain rate and vertical velocity in precipitation.
With analysis of the raw data it is also possible to get measures of snowfall
rate and their related vertical motion which is more variable than rainfall.
Model evaluation of
precipitation is needed to complement cloud evaluation in order to make
sure the right amount of moisture is being removed from the BL by precipitation
and the precipitation has the correct phase (liquid or ice). Similar to clouds,
precipitation has an effect on shortwave (SW) and longwave (LW) radiation. It is therefore important
for the radiation balance, considering its frequent occurrence in the SO.
A dual camera setup was installed on the TAN1802 voyage in order to allow
for stereoscopic determination of cloud base height. This imagery has not been
utilised so far, but a co-authored study has utilised sky camera images collected
on the \textit{Aurora Australis} voyages in comparison with
ceilometer observations \citep{klekociuk2018}.
Co-location with the ceilometer observations provides an
opportunity for algorithm development and cross-validation of results.
Likewise, cloud type and cloud fraction can be determined from sky camera
imagery using a suitable algorithm. A pair of sky cameras could be utilised as
a low-cost instrument for determining cloud base height, cloud fraction
and cloud type on ships of opportunity \citep{klekociuk2018}.

As we identified, ground-based observations have a large additional value
to satellite observations when it comes to BL cloud evaluation. More
ground-based observations in the SO could narrow the spatial and temporal gaps
in our understanding of the region. Therefore, making ship-based atmospheric
observations more accessible is an important task. Currently, deployment
of instruments such as lidar and radars on ships is logistically difficult
and expensive. Smaller and less expensive instruments could mean that they
can be deployed more widely and installed permanently on some ships. Smaller
lidars typically have lower power and inferior noise characteristics,
but a wide coverage could outweigh these deficiencies. Progress in lidar
development could also mean smaller instruments may become equally powerful.
The commercial nature of common instruments means that some instruments
suffer from a number of technical problems and vendors typically show
little interest in resolving these. Open hardware and open software design of
instruments could substantially accelerate progress and collaboration.
Off-the-shelf software defined radio (SDR) receivers and
transceivers\footnote{\url{https://limemicro.com}.} have recently
become widely available. They could provide a basis for development of improved
radars which can utilise many frequencies. Open software for processing
instrument data could mean that implementation of standard techniques and
algorithms for calibration, resampling, noise removal and calculation of derived
products become more available to the scientific community. Likewise,
public sharing of ground-based observational data and building of dataset
collections, similar to the common practice of releasing satellite datasets,
has a potential to accelerate atmospheric research in the SO.
Utilising unusual platforms such as the Argo floats \citep{roemmich2009}
as platforms for atmospheric measurements could provide a vast amount of data and a relatively dense
coverage in the SO.
Air--sea surface fluxes appear to be implicated in the GA7.1/UM11.4 cloud biases
(Chapter 4). Yet, the Coupled Ocean–Atmosphere Response Experiment (COARE)
formulas are potentially not well-tuned for high latitudes and atmospheric
reanalyses exhibit large air--sea flux biases in the SO
\citep{cerovevcki2011}. Evaluation of model air--sea flux biases relative to
reliable in situ flux measurements is needed to make sure they are not
a major cause of the BL cloud biases. Other related biases are in the
representation of marine aerosols, which act as a source of cloud condensation
nuclei (CCN) and ice nucleating particles (INPs) \citep{hartery2020a}.

A reliable comparison of models and observations using a lidar simulator
requires accurate absolute calibration of the observed backscatter
(Chapter 3). We have identified various deficiencies in the vendor-supplied
calibration such as the dead time, afterpulse and overlap calibration in
the MiniMPL. These often result in range-dependent bias of the volume
backscattering coefficient, which may be impossible to reliably correct
for without dedicated calibration measurements. Currently, the ALCF only
supports height-independent calibration utilising observations of liquid
Sc clouds and comparison with a theoretically expected
molecular backscatter (Chapter 4). An automated approach should be developed to calculate
dead time, afterpulse and overlap corrections as an alternative to the 
vendor corrections, which appear to be unreliable. This could lower the
difficulty of utilising ceilometer and lidar data by the wider scientific
community, which is a significant barrier for a wide utilisation of ceilometers
and lidars for model evaluation.
Currently, the ALCF does not support calculation of cross-polarised
backscatter. Adding support for simulation of cross-polarised backscatter
would mean that both channels of the Sigma Space MiniMPL can be utilised
in a comparison with a model and therefore evaluate cloud phase.
Likewise, aerosol and precipitation is currently not taken into account
by the lidar simulator. This not only causes a systematic bias in comparison
with observations, but also prevents the simulator being used for
model aerosol and precipitation evaluation.

A large amount of automatic lidar and ceilometer (ALC) data have been collected by regional and global
networks such as Cloudnet \citep{illingworth2007},
E-PROFILE \citep{illingworth2018}, EARLINET
\citep{pappalardo2014}, ICENET \citep{cazorla2017} and MPLNET
\citep{welton2006}. Most of these networks utilise ceilometers
and lidars already supported by the ALCF such as Vaisala CL31, CL51, Lufft CHM
15k and Sigma Space MiniMPL. These observations haven't been used
extensively for model evaluation. There is a potential to process and compare
these observations with models using the ALCF. Such comparison would complement
past comparisons with satellite observations, and could rival satellite
observations in terms of global spatial and temporal coverage. 

Here, we utilised the Clouds and the Earth's Radiant Energy System (CERES) satellite observations of top of atmosphere (TOA) radiation. Even
though the errors of these observations may be relatively low compared to 
current model cloud biases, systematic errors in the dataset exist due
to unequal diurnal sampling and the need for temporal and angle
interpolation of the raw measurements to calculate daily averages. 
The relatively new instrument National Institute of Standards and Technology
Advanced Radiometer (NISTAR) on the L1-stationed Deep Space Climate Observatory (DSCOVR)
provides a unique viewpoint for Earth radiation observations, from which 
the sunlit part of the Earth, including polar latitudes, is visible continuously.
It would be valuable to compare model radiation fluxes with these new
observations to rule out the effect of systematic biases of CERES on the
model TOA radiation evaluation.

An unresolved question about the origin of the bipolar SW radiation
bias in the high- and low-latitude SO in GA7.1 remains. We know that
the model underestimates low cloud, and at the same time, we identified
some evidence that the optical thickness of individual clouds is overestimated.
This may be the underlying reason for the overestimated reflected TOA SW
radiation in the low-latitude SO.

We identified that BL parametrisation of mass flux can be tuned
to enable Sc cloud simulation in the SO (Chapter 4). However,
a good observational reference for mass flux and the similarity relationships
is lacking in the SO. The BL turbulence and convection
parametrisation is based on large eddy simulations (LES) initialised
from field experiments in the tropical and midlatitude ocean. These may be
inappropriate for the SO. Therefore, new LES initialised from SO field
experiments are needed to make sure these parametrisations are correct in this
region.

We have shown that nudged simulations of GA7.1 provide a very good basis
for identifying model biases compared to ground-based observations. As opposed
to a free-running model, it can be reasonably assumed that any biases are
not the result of a different weather situation simulated by the model
than in reality, and that they are largely due to errors in the subgrid-scale
parametrisation processes not assimilated in the ERA-Interim reanalysis driving
the model or not an input to the model nudging algorithm. As shown in Chapter 4,
a side-by-side comparison of modelled and observed cloud is feasible, and this
can be utilised in future studies of model clouds.

In Chapter 2 we presented a dataset of ground-based observations in the SO.
This dataset provides a unique and comprehensive information on SO atmospheric
conditions and clouds. Work is underway to make this dataset documented and
publicly available. To this end, derived products should be developed in order
to make it easy for the scientific community to use this dataset. In general,
a more concentrated effort is needed to streamline public sharing of atmospheric
observations, especially considering the global and accelerating
effect of climate change, which has been called
"the defining challenge of our time" \citep{wmo2019} by the United Nations
Secretary-General A. Guterres. It is the author's opinion that the
seriousness and urgency of the situation is vastly underestimated even
by the atmospheric science community, which continues to hinder international
cooperation by not sharing data and model code, and by publishing scientific
results in paywalled journals, and therefore putting the well-being of future
generations in jeopardy. A question remains whether the parametrisations
can be formulated in a more physical manner without relying on the similarity
theory relations, which may not be universally applicable globally.

In Chapter 4 we identified that increasing convective mass flux and surface
heat flux can improve Sc cloud simulation. However, it remains
to be proven if mass flux or surface fluxes are underestimated relative to
a physical reference (either an observational reference or large eddy
simulations). More model experiments need to be performed which increase
flux between the turbulently-mixed surface mixed layer and the convective layer,
as well as longer term climate simulation to make sure the modifications
address the actual problem without introducing compensating biases,
and they do not have a negative impact on the
global radiation balance throughout the year. A new BL scheme "CoMorph"
is currently in development at the UK Met Office. Our findings could contribute
to the development of this new scheme, but more experiments need to be performed
with this new scheme and comparison of this new UK Met Office scheme with the ground-based
SO observations should be performed.

\clearpage
